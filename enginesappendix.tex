\chapter{Nestable Engines}\label{nestable-engines}

Our implementation of ferns in Chapter~\ref{fernsimpl} requires
nestable engines \cite{RDybvi89,hieb94subcontinuations}, which we
present here with minimal comment.  The implementation uses a global
variable, \scheme|state|, which holds two values: the number of ticks
available to the currently running engine or \scheme|#f| representing
infinity; and a continuation. \scheme|make-engine| makes an engine out
of a thunk.  \scheme|engine| is a macro that makes an engine from an
expression.  \scheme|timed-lambda| is like \scheme|lambda| except that
it passes its body as a thunk to \scheme|expend-tick-to-call|, which
ensures a tick is spent before the body is evaluated and passes the
suspended body to the continuation if no ticks are available. Programs
that use this embedding of nestable engines (and by extension our
embedding of \scheme|frons|) should not use \scheme|call/cc|, because
the uses of \scheme|call/cc| in the nestable engines implementation
may interact with other uses in ways that are difficult for the
programmer to predict.

%\newpage
%\enlargethispage{30pt}
\schemedisplayspace
\schemeinput{fernscode/coaxappendix.ss}

\schemeinput{fernscode/appendix-extra.ss}
