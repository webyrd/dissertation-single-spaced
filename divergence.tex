\chapter{A Slight Divergence}\label{divergencechapter}

In this chapter we explore the divergence of relational programs.  We
present several divergent miniKanren programs; for each program we
consider different techniques that can be used to make the program
terminate.

By their very nature, relational programs are prone to divergence.  As
relational programmers, we may ask for an infinite number of answers
from a program, or we may look for a non-existent answer in an
infinite search tree.  In fact, miniKanren programs can (and do!)
diverge for a variety of reasons.  A frustration common to beginning
miniKanren programmers is that of carefully writing or deriving a
program, only to have it diverge on even simple test cases.  Learning
to recognize the sources of divergence in a program, and which
techniques can be used to achieve termination, is a critical stage in
the evolution of every relational programmer.

To help miniKanren programmers write relations that terminate, this
chapter presents several divergent example programs; for each program,
we discuss why it diverges, and how the divergence can be avoided.

It is important to remember that a single relational program may
contain multiple, and completely different, causes of divergence; such
programs may require a variety of techniques in order to
terminate\footnote{Challenge for the reader: construct a single
miniKanren program that contains every cause of divergence discussed
in this chapter.  Then use the techniques from this chapter to ``fix''
the program.}.  Also, a single technique may be useful for avoiding
multiple causes of divergence, as will be made clear in the examples
below.  miniKanren does not currently support all of these
techniques (such as operators on cyclic terms)---unsupported
techniques are clearly identified in the text.  Even techniques not
yet supported by miniKanren are of value, however, since they may be
supported by other programming languages.

We now present the divergent example programs, along with techniques
for avoiding divergence.  


\section*{Example 1}

Consider the divergent \scheme|run*| expression

\schemedisplayspace
\begin{schemedisplay}
(run* (q)
  (exist (x y z)
    (pluso x y z)
    (== `(,x ,y ,z) q)))
\end{schemedisplay}

\noindent where \scheme|pluso| is the ternary addition relation defined
in Chapter~\ref{arithchapter}.  This expression diverges because
\mbox{\scheme|(pluso x y z)|} succeeds an unbounded number of times;
therefore, the \scheme|run*| never stops producing answers.  Although
it could be argued that this is a ``good'' infinite loop, and that we
got what we asked for, presumably we want to see some of these
answers.  Also, the user has no way of knowing that the system is
producing any answers, since the divergence might be due to one of the
other causes described below.  (Not to mention that, in general, the
user cannot tell whether the program is diverging or merely taking a
very long time to produce an answer.)

We can avoid this divergence in several different ways:

\begin{enumerate}

\item We could replace the \scheme|run*| with \scheme|run|$^n$, where
$n$ is some positive integer.  This will return the $n$ answers,
although miniKanren's interleaving search makes the order in which
answers are produced difficult to predict.

\item Instead of using the \scheme|run| interface, we could directly
manipulate the answer stream passed as the second argument to
\scheme|take| (Chapter~\ref{mkimplchapter}), and examine the answers
one at a time.  This the ``read-eval-print loop'' approach is used by
Prolog systems, and is trivial to implement in miniKanren by redefining 
\scheme|take|.

\item We can use \scheme|onceo| or \scheme|condu| to ensure that goals
that might succeed an unbounded number of times succeed only once.  Of
course, these operators are non-declarative, so we reject this
approach.  Instead, it would be better to use a \scheme|run1|.

\item A more sophisticated approach is to represent infinitely many
answers as a single answer by using constraints.  For example, one way
to express that \scheme|x| is a natural number other than $2$ is to
associate \scheme|x| with $0, 1, 3, \ldots$.  Clearly, there are
infinitely many such associations, and enumerating them can lead to an
unbounded number of answers.  Instead, we might represent the same
information using the single disequality constraint \mbox{\scheme|(=/= 2 x)|}.

Similarly, we might use a clever data representation rather than a
constraint to represent infinitely many answers as a single term.  For
example, using the little-endian binary representation of natural
numbers presented in Chapter~\ref{arithchapter}, the term
\mbox{\scheme|`(1 . ,x)|} represents any one of the infinitely many
odd naturals.

Using this technique, programs that previously produced infinitely
many answers may fail finitely, proving that no more answers exist.
Unfortunately, it is not always possible to find a constraint or data
representation to concisely represent infinitely many terms.  For
example, although the data representation from
Chapter~\ref{arithchapter} makes it easy to express every odd natural
as a single term, there is no little-endian binary list that succinctly
represents every prime number.  Similarly, disequality constraints are
not sufficient to concisely express that some term does not appear in
an uninstantiated tree\footnote{However, the freshness
constraint (\scheme|hash|) described in Chapter~\ref{akchapter} allows
us to express a similar constraint.}.

\end{enumerate}



\section*{Example 2}

Consider the divergent \scheme|run1| expression

\wspace

\mbox{\scheme|(run1 (q) (==-no-check `(,q) q))|}

\wspace

\noindent The unification of \scheme|q| with \scheme|`(,q)| results in
a substitution containing a circularity\footnote{The \scheme|=/=-no-check|
disequality operator (Chapter~\ref{diseqchapter}) suffers from the
same problem, since it can add circularities to the constraint
store.}: \mbox{\scheme|`((q . (,q)))|}.  However, it is not
unification that diverges, or subsequent calls to \scheme|walk|.
Rather, the reification of \scheme|q| at the end of the computation
calls \scheme|walk*| (Chapter~\ref{mkimplchapter}), which
diverges\footnote{The non-logical operator \scheme|project| also calls
\scheme|walk*|, and can therefore diverge on circular substitutions.}.

We can avoid this divergence in several different ways:

\begin{enumerate}

\item We can use \scheme|==| rather than \scheme|==-no-check| to
  perform sound unification with the occurs check.  The goal
  \mbox{\scheme|(== `(,q) q)|} violates the occurs check and therefore
  fails; hence, \mbox{\scheme|(run1 (q) (== `(,q) q))|} returns
  \mbox{\schemeresult|`()|} rather than diverging\footnote{Similarly,
    we can use \scheme|=/=| rather than \scheme|=/=-no-check| when
    introducing disequality constraints.}.  Since the occurs check can
  be expensive, we may wish to restrict \scheme|==| to only those
  unifications that might introduce a circularity, such as in the
  application line of a type inferencer; this requires reasoning about
  the program.  Alternatively, we can always be safe by using only
  \scheme|==| rather than \scheme|==-no-check|\footnote{As pointed out
    by~\citet{occurcheck} this approach may be overly conservative.
    However, since our primary interest is in avoiding divergence, this
    approach seems reasonable.}.

\item Since the reification of \scheme|q| causes divergence in
this example, the \scheme|run| expression will terminate if we do not
reify the variable associated with the circularity.  For example,

\mbox{\scheme|(run1 (q) (exist (x) (==-no-check `(,x) x)))|}

\noindent returns \mbox{\schemeresult|`(_.0)|}.  Although the
\scheme|run| expression terminates, the resulting substitution is
still circular: \mbox{\scheme|`((,x . (,x)))|}.
However, unless we allow infinite terms,
the unification \mbox{\scheme|(==-no-check `(,x) x)|} is \emph{unsound}.  This
is a problem for the type inferencers based on the simply typed $\lambda$-calculus, 
for example, since self-applications such
as \mbox{\scheme|(f f)|} should not type check (see the inferencer in section~\ref{aktypeinf}).  
If we do not perform
the occurs check, and the circular term
is not reified, the type inference will succeed instead of failing.
Clearly this is not an acceptable way to avoid divergence.  However,
it is important to understand why the program above terminates,
since it is possible to unintentionally write programs that abuse
unsound unification, unless we use \scheme|==| everywhere.

\item Since reification is the cause of divergence in this
example, we can just avoid reification entirely and return the raw
substitution.  The user must determine which associations in the
substitution are of interest; furthermore, the user must check the
substitution for circularities introduced by unsound unification.
There is one more problem with both this approach and the previous
one: the occurs check can prevent divergence by making the program
fail early, which may avoid an unbounded number of successes or a
futile search for a non-existent answer in an infinite search space.

\item Another approach to avoiding divergence is to allow infinite (or
  \emph{cyclic}) terms, as introduced by Prolog
  II~\cite{prologtenfigs,DBLP:conf/fgcs/Colmerauer84,Colmerauer82}.
  Then the unification \mbox{\scheme|(==-no-check `(,q) q)|} is sound,
  even though it returns a circular substitution.  miniKanren does not
  currently support infinite terms; however, it would not be difficult
  to extend the reifier to handle cyclic terms, just as many Scheme
  implementations can print circular lists.

\end{enumerate}




\section*{Example 3}

Consider the divergent \scheme|run1| expression\footnote{Recall that
\scheme|alwayso| was defined in Chapter~\ref{mkintrochapter} as
\mbox{\scheme|(define alwayso (anyo (== #f #f)))|}.  However, for the
purposes of this chapter we define \scheme|alwayso| as

\begin{schemedisplay}
(define alwayso
  (letrec ((alwayso (lambda ()
                      (conde
                        ((== #f #f))
                        ((alwayso))))))
    (alwayso)))
\end{schemedisplay}

\noindent This is because tabling (Chapters~\ref{tablingchapter} and
\ref{tablingimplchapter}) uses reification to determine if a call is a
variant of a previously tabled call.  Since all procedures have the
same reified form (\schemeresult|#<procedure>| under Chez Scheme, for
example), and since \scheme|anyo| takes a goal (a procedure) as its
argument, tabling \scheme|anyo| can lead to unsound behavior.}

\wspace

\mbox{\scheme|(run1 (q) alwayso fail)|}

\wspace

\noindent where \scheme|fail| is defined as \mbox{\scheme|(== #t #f)|}.
Recall that the body of a \scheme|run| is an implicit
conjunction\footnote{\mbox{\scheme|(run1 (q) g1 g2)|} expands into an
expression containing \mbox{\scheme|(exist () g1 g2)|}.}.  In order for
the \scheme|run| expression to succeed, both \scheme|alwayso| and
\scheme|fail| must succeed.  First, \scheme|alwayso| succeeds, then
\scheme|fail| fails.  We then backtrack into \scheme|alwayso|, which
succeeds again, followed once again by failure of the \scheme|fail|
goal.  Since \scheme|alwayso| succeeds an unbounded number of times,
we repeat the cycle forever, resulting in divergence.


We can avoid this divergence in several different ways:

\begin{enumerate}

\item We could simply reorder the goals: \mbox{\scheme|(run1 (q) fail alwayso)|}.
This expression returns \mbox{\schemeresult|`()|} rather
than diverging, since \scheme|fail| fails before \scheme|alwayso| is
even tried.  miniKanren's conjunction operator (\scheme|exist|) is
commutative, but only if an answer exists.  If no answer exists, then
reordering goals within an \scheme|exist| may result in divergence
rather than failure\footnote{We say that conjunction is commutative,
  \emph{modulo divergence versus failure}.}.

However, reordering goals has its disadvantages.  For many programs,
no ordering of goals will result in finite failure (see the remaining
example in this chapter).  Also, by committing to a certain goal
ordering we are giving up on the declarative nature of relational
programming: we are specifying \emph{how} the program computes, rather
than only \emph{what} it computes.  For these reasons, we should
consider alternative solutions.

\item We may be able to use constraints or clever data structures to
represent infinitely many terms as a single term (as described in
Example~1).  If we can use these techniques to make all the conjuncts
succeed finitely many times, then the program will terminate
regardless of goal ordering.

\item Another approach to making the conjuncts succeed finitely many
times is to use tabling, described in Chapter~\ref{tablingchapter}.
Tabling is a form of memoization---we remember every distinct call to
the tabled goal, along with the answers produced.  When a tabled goal
is called, we check whether the goal has previously been called with
similar arguments---if so, we use the tabled answers.

In addition to potentially making goals more efficient by avoiding
duplicate work, tabling can improve termination behavior by cutting
off infinite recursions.  For example, the tabled version of
\scheme|alwayso| succeeds exactly once rather than an unbounded number
of times.  Therefore, \mbox{\scheme|(run1 (q) alwayso fail)|}
returns \mbox{\schemeresult|`()|} rather than diverging when
\scheme|alwayso| is tabled.

Unfortunately, tabling has a major disadvantage: it does not work if
one or more of the arguments to a tabled goal changes with each
recursive call\footnote{As demonstrated by the \scheme|geno| example
in a later footnote.}.

\item We could perform a \emph{dependency analysis} on the
conjuncts---if the goals do not share any logic variables, they cannot
affect each other.  Therefore we can run the goals in parallel,
passing the original substitution to each goal.  If either goal fails,
the entire conjunction fails.  If both goals succeed, we take the
Cartesian product of answers from the goals, and use those new
associations to extend the original substitution.

miniKanren does not currently support this technique; however,
miniKanren's interleaving search should make it straightforward to run
conjuncts in parallel.  A run-time dependency analysis would also be
easy to implement\footnote{Ciao Prolog~\cite{Hermenegildo95strictand}
  performs dependency analysis of conjuncts, along with many other
  analyses, to support efficient parallel logic programming.}.

\item We could address the problem directly by trying to make our
conjunction operator commutative.  For example, we could run both goal
orderings in parallel\footnote{We might
do this by wrapping the goals in a
fern (Chapter~\ref{fernschapter}).}, \mbox{\scheme|(exist () alwayso fail)|} and
\mbox{\scheme|(exist () alwayso fail)|}, and see if either ordering
converges.  If so, we could commit to this goal ordering.
Unfortunately, this commitment may be premature, since the goal
ordering we picked might diverge when we ask for a second answer,
while the other ordering may fail finitely after producing a single
answer.

We could try \emph{all} possible goal orderings, but this is
prohibitively expensive for all but the simplest programs.  In
particular, recursive goals containing conjunctions will result in an
exponential explosion in the number of orderings.

For these reasons, miniKanren does not currently provide a
commutative conjunction operator.  However, future versions of
miniKanren may include an operator that \emph{simulates} full
commutative conjunction using a combination of tabling, parallel goal
evaluation, and continuations (see the Future Work chapter).

\end{enumerate}


\section*{Example 4}

Consider the \scheme|run1| expression \mbox{\scheme|(run1 (x) (pluso bn2 x bn1))|}.
If \scheme|pluso| represents the ternary addition relation
over natural numbers, there is no value for \scheme|q| that satisfies
\mbox{\scheme|(pluso bn2 x bn1)|} (since $2 + x = 1$ has no solution in
the naturals).  Ideally, the \scheme|run1| expression will return
\mbox{\schemeresult|`()|}.  However, a naive implementation of
\scheme|pluso| that enumerates values for \scheme|x| will diverge,
since it will keep associating \scheme|x| with larger numbers without
bound.  Since \scheme|x| grows with each recursive call, tabling
\scheme|pluso| will not help.

We can avoid this divergence in several different ways:

\begin{enumerate}

\item We can relax the domain of \scheme|x| to include negative
integers---then the \scheme|run1| expression will return
\mbox{\schemeresult|`(bnminus1)|}.  However, changing \scheme|run1| to
\scheme|run2| still results in divergence, since $2 + x = 1$ has only
a single solution in the integers.

\item We could use a domain-specific constraint system.  For example,
instead of writing an addition goal, we could use Constraint Logic
Programming over the integers (also known as ``CLP(Z)'').  If we restrict
the sizes of our numbers, we could use CLP(FD) (Constraint Logic
Programming over finite domains).

Alas, no single constraint system can express every interesting
relation in a non-trivial application.  We could try to create a
custom constraint system for each application we write, but this may
be a very difficult task, especially since constraints may interact
with other language features in complex ways.

miniKanren currently supports four kinds of constraints: unification
and disunification constraints using \scheme|==| and
\scheme|=/=| (Chapters~\ref{mkintrochapter} and \ref{diseqchapter});
$\alpha$-equivalence constraints using nominal
unification (Chapter~\ref{akchapter}); and freshness constraints using
\scheme|hash| (Chapter~\ref{akchapter})\footnote{Some non-published
versions of miniKanren have also supported \scheme|pa/ir| constraints:
\mbox{\scheme|(pa/ir x)|} expresses that \scheme|x| can never be
instantiated as a pair.  Uses of \scheme|pa/ir| can typically be
removed through careful use of tagging, however, so we do not include
the constraint in this dissertation.}.  Future versions of miniKanren
will likely support more sophisticated constraints.

\item Another approach is to bound the size of the terms in the
recursive calls to \scheme|pluso|.  For example, if we represent numbers
as binary lists, we know that the lengths of the first two arguments
to \scheme|pluso| (the summands) should never exceed the length of the
third argument (the sum).  By encoding these bounds on term size in
our \scheme|pluso| relation, the call \mbox{\scheme|(pluso bn2 x bn1)|}
will fail finitely.  We use exactly this technique when defining
\scheme|pluso| in Chapter~\ref{arithchapter}.

Bounding term sizes is a very powerful technique, as is demonstrated
in the relational arithmetic chapter of this dissertation.  But as
with the other techniques presented in this chapter, it has its
limitations.  Establishing relationships between argument sizes may
require considerable insight into the relation being expressed.  In
fact, the arithmetic definitions in Chapter~\ref{arithchapter},
including the bounds on term size, were derived from mathematical
equations; this code would be almost impossible to write otherwise\footnote{For example, see the definition of \scheme|logo| in section~\ref{arithexplog}.}.

Furthermore, overly-eager bounds on term size can themselves cause
divergence.  For example, assume that we know arguments \scheme|x| and
\scheme|y| represent lists, which must be of the same length.  We
might be tempted to first determine the length of \scheme|x|, then
determine the length of \scheme|y|, and finally compare the result.
However, if \scheme|x| is an unassociated logic variable, it has no
fixed length: we could cdr down \scheme|x| forever, inadvertently
lengthening \scheme|x| as we go.  Instead, we must
\emph{simultaneously} compare the lengths of \scheme|x| and
\scheme|y|.  To make the task more difficult, we want to enforce the
bounds while we are performing the primary computation of the relation
(for example, while performing addition in the case of \scheme|pluso|).
In fact, lazily enforcing complex bounds between multiple arguments is
likely to be more difficult than writing the underlying relation.

Another problem with bounds on term sizes is that they may not help
when arguments share logic variables.  For example, consider the
\scheme|lesslo| relation: \mbox{\scheme|(lesslo x y)|} succeeds if
\scheme|x| and \scheme|y| are lists, and \scheme|y| is longer than
\scheme|x|.  We can easily implement \scheme|lesslo| by simultaneously
cdring down \scheme|x| and \scheme|y|:

\schemedisplayspace
\begin{schemebox}
(define lesslo
  (lambda-e (x y)
    (`(() (_ . _)))
    (`((_ . ,xd) (_ . ,yd))
     (lesslo xd yd))))
\end{schemebox}

However, consider the call \mbox{\scheme|(lesslo x x)|}.  The
first \scheme|lambda-e| clause fails, while the second clause results
in a recursive call where both arguments are the same uninstantiated
variable.  Therefore \mbox{\scheme|(lesslo x x)|} diverges.

If we were to table \scheme|lesslo|, \mbox{\scheme|(lesslo x x)|}
would fail instead of diverging.  Unfortunately, sharing of arguments
in more complicated relations may result in arguments growing with
each recursive call, which would defeat tabling.

\end{enumerate}

In this section we have examined several divergent miniKanren
programs, investigated the causes of their
divergence\footnote{miniKanren's interleaving search avoids some forms
  of divergence that afflict Prolog, which uses an incomplete search
  strategy equivalent to depth-first search.  For example, the
  left-recursive \scheme|swappendo| relation from
  Chapter~\ref{mkintrochapter} is equivalent to the standard
  \scheme|appendo| relation in miniKanren.  In Prolog, however,
  \scheme|swappendo| diverges in many cases that \scheme|appendo|
  terminates, even when answers exist.  (Although tabling can be used
  to avoid divergence for left-recursive Prolog goals---indeed, this
  is one of the main reasons for including tabling in a Prolog
  implementation.)}, and considered techniques we can use to make
these programs converge.  As miniKanren programmers, divergence, and
how to avoid it, should never be far from our minds.  Indeed, every
extension to the core miniKanren language can be viewed as a new
technique for avoiding divergence\footnote{For example, the freshness
  constraints of nominal logic allow us to express that a nom
  \scheme|a| does not occur free within a variable \scheme|x|.
  Without such a constraint, we would need to instantiate \scheme|x|
  to a potentially unbounded number of ground terms to establish that
  \scheme|a| does not appear in the term.}.

In the next chapter we present a relational arithmetic system that
uses bounds on term size to establish strong termination guarantees.


% [examples that converge in miniKanren, but might diverge in Prolog, other LP systems, or simplified variants of mk]

% \schemedisplayspace
% \begin{schemedisplay}
% (define appendo
%   (lambda (l s out)
%     (conde
%       ((exist (a d res)
%          (== `(,a . ,d) l)
%          (== `(,a . ,res) out)
%          (appendo d s res)))
%       ((== '() l) (== s out)))))
% \end{schemedisplay}

% \schemedisplayspace
% \begin{schemedisplay}
% (define appendo
%   (lambda (l s out)
%     (conde
%       ((== '() l) (== s out))
%       ((exist (a d res)
%          (== `(,a . ,d) l)
%          (== `(,a . ,res) out)
%          (appendo d s res))))))
% \end{schemedisplay}



% \schemedisplayspace
% \begin{schemedisplay}
% (define geno
%   (lambda (n x)
%     (conde
%       ((== n x))
%       ((geno (add1 n) x)))))

% (run1 (q)
%   (exist (x y)
%     (geno 0 x)
%     (geno 0 y)
%     (== 1 x)))
% \end{schemedisplay}
