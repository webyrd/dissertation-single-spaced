\chapter{Techniques III: Tabling}\label{tablingchapter}

This chapter introduces \emph{tabling}, an extension of memoization to
logic programming.  We present a full implementation of tabling for
miniKanren in Chapter~\ref{tablingimplchapter}.

This chapter is organized as follows.  In section~\ref{tablingmemo} we
review memoization as used in functional programming.
Section~\ref{tablingintro} introduces tabling, explains how tabling
differs from memoization, and describes a few of the many applications
of tabling.  In section~\ref{tablingform} we present the
\scheme|tabled| form, used to create tabled relations.  In
section~\ref{tablingexamples} we examine several examples of tabled
relations, and in section~\ref{tablinglimitations} we discuss the
limitations of tabling.

\section{Memoization}\label{tablingmemo}

Consider the naive Scheme implementation of the Fibonacci function.

\schemedisplayspace
\begin{schemedisplay}
(define fib
  (lambda (n)
    (cond
      ((= 0 n) 0)
      ((= 1 n) 1)
      (else (+ (fib (- n 1)) (fib (- n 2)))))))
\end{schemedisplay}

\noindent 
The call \mbox{\scheme|(fib 5)|} results in calls to
\mbox{\scheme|(fib 4)|} and \mbox{\scheme|(fib 3)|}; the resulting
call to \mbox{\scheme|(fib 4)|} also calls \mbox{\scheme|(fib 3)|}.
The call \mbox{\scheme|(fib 5)|} therefore results in two calls to
\mbox{\scheme|(fib 3)|}, the second of which performs duplicate work.
Similarly, \mbox{\scheme|(fib 5)|} results in three calls to
\mbox{\scheme|(fib 2)|}, five calls to \mbox{\scheme|(fib 1)|}, and
three calls to \mbox{\scheme|(fib 0)|}.  Due to these redundant calls,
the time complexity of \scheme|fib| is exponential in $n$.

To avoid this duplicate work, we could record each distinct call to
\scheme|fib| in a table, along with the answer returned by that call.
Whenever a duplicate call to \scheme|fib| is made, \scheme|fib| would
return the answer stored in the table instead of recomputing the
result.  This optimization technique, known as \emph{memoization}
\cite{Michie68}, can result in a lower complexity class for the
running time of the memoized function.  Indeed, the memoized version
of \scheme|fib| runs in linear rather than exponential time.

Memoization is a common technique in functional programming, since it
often improves performance of recursive functions.  In this chapter we
consider the related technique of \scheme|tabling|, which generalizes
memoization to logic programming.

\section{Tabling}\label{tablingintro}

Tabling is a generalization of memoization; tabling allows a relation
to store and reuse its previously computed results.  Tabling a
relation is more complicated than memoizing a function, since a
relation returns a potentially infinite stream of substitutions rather
than a single value.  Also, the arguments to a tabled relation can
contain unassociated logic variables or partially instantiated terms,
which complicates determining whether a call is a variant of a
previously seen call.

Tabling, like memoization, can result in dramatic performance gains
for some programs.  For example, combining tabling with Prolog's Definite
Clause Grammars~\cite{dcg86} makes it trivial to write efficient
recursive descent parsers that handle left-recursion\footnote{One
  important use of tabling by Prolog systems is to handle
  left-recursive definitions of goals; due to Prolog's incomplete
  depth-first search, calls to left-recursive goals often diverge.
  Since miniKanren uses a complete search strategy, handling
  left-recursion is not a problem.  However, we will see in
  section~\ref{tablingexamples} that there are other programs we want
  to write that terminate under tabling but diverge
  otherwise.}~\cite{DBLP:conf/padl/BecketS08}---these parsers are
equivalent to ``packrat'' parsing~\cite{packrat02}.  Tabling is also
useful for writing programs that must calculate fixed points, such as
abstract interpreters and model checkers~\cite{memoingforlp,dra09}.
However, the real reason we are interested in tabling is that many
relations that would otherwise diverge terminate under tabling, as we
will see in section~\ref{tablingexamples}.

An excellent introduction to tabling and its uses is Warren's survey~\cite{memoingforlp}.

\section{The {\bf tabled} Form}\label{tablingform}

Tabled relations are constructed using the \scheme|tabled| form:

\wspace

\mbox{\scheme|(tabled (x ...) g g* ...)|}

\wspace

\noindent For example, 

\wspace

\mbox{\scheme|(define fo (tabled (z) (== z 5)))|}

\wspace

\noindent defines a top-level tabled goal constructor named
\scheme|fo|.  Each tabled goal constructor has its own local table,
which can be garbage collected once there are no live references to
the goal constructor.  Keep in mind that the table is associated with
the goal constructor, not the goal returned by the goal constructor.

Calls to a tabled relation come in two flavors: master calls and slave
calls.  A \emph{master call} is a call to a tabled relation whose
arguments are not (yet) stored in the table.  A \emph{slave call} is a
call whose arguments are found in the table; each slave call is a
\emph{variant} of some master call.

Two calls to the same tabled relation are variants of each other if
their arguments are the same, up to consistent renaming of
unassociated logic variables\footnote{In other words, the two lists of arguments
  to the relation, when reified with respect to their ``current'' substitutions, 
  must be \scheme|equal?|.}. For example, consider the calls
\mbox{\scheme|(mulo y z 5)|} and \mbox{\scheme|(mulo w w x)|} in the
substitutions \mbox{\scheme|`((y . z))|} and \mbox{\scheme|`((x . 5))|},
respectively.  Taking the substitutions into account, these calls are equivalent to
\mbox{\scheme|(mulo z z 5)|} and \mbox{\scheme|(mulo w w 5)|}, which are variants of 
each other.  However, the calls \mbox{\scheme|(mulo w w x)|} and \mbox{\scheme|(mulo y z z)|}
are variants only if \scheme|w| is associated with \scheme|x|, and \scheme|y| is associated with \scheme|z|,
respectively.  For the same reason, 
\mbox{\scheme|(mulo w 5 6)|} and \mbox{\scheme|(mulo y z 6)|} are variants only if 
\scheme|z| is associated with 5 in the substitution in place for the second call.

\section{Tabling Examples}\label{tablingexamples}

We are now ready to examine examples of tabled relations.  The
canonical example relation, \scheme|patho|\footnote{The path examples
  in this section are taken from~\citet{memoingforlp}.}, finds all
paths between two nodes in a directed graph.  The goal
\mbox{\scheme|(patho x y)|} succeeds if there is a directed edge from
\scheme|x| to \scheme|y|, or if there is an edge from \scheme|x| to
some node \scheme|z| and there is a path from \scheme|z| to
\scheme|y|.

\schemedisplayspace
\begin{schemedisplay}
(define patho
  (lambda (x y)
    (conde
      ((arco x y))
      ((exist (z)
         (arco x z)
         (patho z y))))))
\end{schemedisplay}

\noindent The goal \mbox{\scheme|(arco x y)|} succeeds if there is a
directed edge from node \scheme|x| to node \scheme|y|.

\schemedisplayspace
\begin{schemedisplay}
(define arco
  (lambda (x y)
    (conde
      ((== 'a x) (== 'b y))
      ((== 'c x) (== 'b y))
      ((== 'b x) (== 'd y)))))
\end{schemedisplay}

\noindent This definition of \scheme|arco| represents edges from
\scheme|'a| to \scheme|'b|, \scheme|'c| to \scheme|'b|, and
\scheme|'b| to \scheme|'d|.

\noindent The expression \mbox{\scheme|(run* (q) (patho 'a q))|}
returns \mbox{\scheme|'(b d)|}, indicating that only the nodes
\scheme|'b| and \scheme|'d| are reachable from \scheme|'a|.

Now let us redefine \scheme|arco| to represent a different set of
directed edges, this time with a circularity between nodes \scheme|'a|
and \scheme|'b|.

\newpage

\schemedisplayspace \begin{schemedisplay}
(define arco
  (lambda (x y)
    (conde
      ((== 'a x) (== 'b y))
      ((== 'b x) (== 'a y))
      ((== 'b x) (== 'd y)))))
\end{schemedisplay}

\noindent Using the new definition of \scheme|arco|, the expression
\mbox{\scheme|(run* (q) (patho 'a q))|} now diverges.  We can
understand the cause of this divergence if we replace \scheme|run*|
with \scheme|run10|.

\wspace

\noindent\scheme|(run10 (q) (patho 'a q))|  $\Rightarrow$ \scheme|'(b a d b a d b a d b)|

\wspace

\noindent Because of the circular path between \scheme|'a| and
\scheme|'b|, \mbox{\scheme|(patho 'a q)|} keeps finding longer and
longer paths between \scheme|'a| and the nodes \scheme|'b|, \scheme|'a|,
and \scheme|'d|.  To avoid this problem, we can table
\scheme|patho|.

\schemedisplayspace
\begin{schemedisplay}
(define patho
  (tabled (x y)
    (conde
      ((arco x y))
      ((exist (z)
         (arco x z)
         (patho z y))))))
\end{schemedisplay}

\noindent \mbox{\scheme|(run* (q) (patho 'a q))|} then converges,
returning \mbox{\scheme|'(b a d)|}.

Now let us consider a mutually recursive program.

\schemedisplayspace
\begin{schemedisplay}
(letrec ((fo (lambda (x)
               (conde
                 ((== 0 x))
                 ((go x)))))
         (go (lambda (x)
               (conde
                 ((== 1 x))
                 ((fo x))))))
  (run* (q) (fo q)))
\end{schemedisplay}

\noindent This expression diverges.  If we replace
\scheme|run*| with \scheme|run10| the program converges with the value
\mbox{\scheme|'(0 1 0 1 0 1 0 1 0 1)|}.  If we table either
\scheme|fo|, \scheme|go|, or both, \mbox{\scheme|(run* (q) (fo q))|}
converges with the value \mbox{\scheme|'(0 1)|}.


\section{Limitations of Tabling}\label{tablinglimitations}

Tabling is a remarkably useful addition to miniKanren, and can be
used to improve efficiency of relations and
(sometimes) avoid divergence.  Unfortunately, tabling is not a
panacea.  In fact, tabling can be trivially defeated by changing one
or more arguments in each call to a tabled relation.  For example,
consider the ternary multiplication relation \scheme|mulo| from
Chapter~\ref{arithchapter}.  The arguments in the call

\wspace

\noindent
\mbox{\scheme|(mulo `(1 1 . ,x) x `(0 0 0 1 . ,x))|}\footnote{This example is due to Oleg
  Kiselyov (personal communication).}

\wspace

\noindent
all share the variable \scheme|x|.
The resulting goal succeeds only if there exists a non-negative
integer $x$ that satisfies $(3 + 4x) \cdot x = 8 + 16x$.  \scheme|mulo| 
enumerates all non-negative integer values for \scheme|x| until it
finds one that satisfies this equation.  However, if no such
\scheme|x| exists the call to \scheme|mulo| will diverge.  Tabling
will not help, since the value of \scheme|x| keeps changing.

Another disadvantage of tabling is that it can greatly increase the
memory consumption of a program.  This is a problem with memoization
in general.  For example, consider the tail-recursive
accumulator-passing-style Scheme definition of factorial\footnote{The
  call \mbox{\scheme|(!-aps n 1)|} calculates the factorial of
  \scheme|n|.}.

\schemedisplayspace
\begin{schemedisplay}
(define !-aps
  (lambda (n a)
    (cond
      ((zero? n) a)
      (else (!-aps (sub1 n) (* n a))))))
\end{schemedisplay}

\noindent Other than the space used to represent numbers, this
function uses a bounded amount of memory\footnote{Scheme
  implementations are required to handle tail calls properly---thus
  \scheme|!-aps| uses a constant amount of stack space}.  However, the
memoized version of \scheme|!-aps| uses an unbounded amount of memory
if \scheme|n| is negative, and otherwise uses an amount of memory
linear in \scheme|n|.

Chapter~\ref{tablingimplchapter} presents a complete implementation of
tabling for miniKanren; this implementation has several limitations.
The first limitation is that tabled relations must be closed; a tabled
goal constructor cannot contain free logic variables, since
associations for those variables would be thrown away.  This is a
consequence of not storing entire substitutions in a relation's table,
as described in section~\ref{tablingrep}.

Another limitation is that arguments passed to tabled relations must
be ``printable'' (or ``reifiable'') values.  For example, tabled
relations should never be passed functions, including goals, since all
functions reify to the same value\footnote{Pure relations should never
take functions as arguments anyway, since miniKanren does not support
higher-order unification, and cannot meaningfully construct functions
when running backwards.}.

The most significant limitation of our tabling implementation is that
it does not currently support disequality constraints, nominal
unification, or freshness constraints.  How to best combine tabling
and constraints is an open research problem~\cite{TCHR08}.
